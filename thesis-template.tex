\documentclass[a4paper,11pt,fonts,numbering,twoside,margins,listingsleft]{./tulpackage/tulthesis}

% Správa bibliografie s biblatex-biber
\usepackage[backend=biber]{biblatex}
% Cesta k bibliografii
\addbibresource{./bibliography/references.bib}

% Nastavení jazyka dokumentu na češtinu
\setdefaultlanguage{czech}

% Zvýšení řádkování pro lepší čitelnost návodu
\linespread{1.5}

% Vytvoření indexu dokumentu
\makeindex

% Vlastní seznamy s enumitem
\newlist{itemize*}{itemize}{2}
\setlist[itemize*]{itemsep=3pt, parsep=3pt, label=$\bullet$}
\setlist[itemize*,2]{label={--}}

% Nastavení sazby a zalamování textu
\sloppy
\lefthyphenmin=2
\righthyphenmin=3
\hyphenpenalty=5000
\tolerance=1500
\emergencystretch=3.5em
\hbadness=2000

% Penalizace osamocených řádků
\widowpenalty=10000
\clubpenalty=10000

% Definice pomocného příkazu pro ukázky
\newcommand{\cmddemo}[1]{\bigskip\parbox[c]{3.9cm}{\cmd{#1}}\parbox[c]{10cm}{\csname #1\endcsname}\bigskip}

% Příkazy specifické pro tento dokument
\newcommand{\argument}[1]{{\ttfamily\color{\tulcolor}#1}}
\newcommand{\argumentprom}[1]{\argument{\{\emph{#1}\}}}
\newcommand{\argumentindex}[1]{\argument{#1}\index{#1}}
\newcommand{\prostredi}[1]{\argumentindex{#1}}
\newcommand{\prikazneindex}[1]{\argument{\textbackslash{}#1}}
\newcommand{\prikaz}[1]{\prikazneindex{#1}\index{#1@\textbackslash #1}}
\newcommand{\polozka}[1]{\item[#1]\mbox{}\\}

% Deklarace pro titulní stránku (DOPLŇ)
\TULtitle{Název práce česky}{Thesis title in English}
\TULauthor{Jméno Příjmení}

% Nastavení pro bakalářské, diplomové a disertační práce (DOPLŇ)
\TULprogramme{KÓD}{Studijní program česky}{Study programme in English}
\TULbranch{KÓD}{Studijní obor/specializace česky}{Study branch/specialization in English}
\TULsupervisor{Titul Jméno Příjmení}
% \TULconsultant{Titul Jméno Příjmení} % pokud máš
\TULyear{2026}

% Nastavení pro habilitační práce (odkomentováno pouze při potřebě)
% \TULbranch{}{Technická kybernetika}{Technical cybernetics}
% \TULyear{2021}

\begin{document}

% Úvodní stránky (titulní list, zadání, prohlášení) – generované třídou
\ThesisStart{male} % nebo female
% \ThesisStart{zahajeni.pdf} % pokud vkládáš STAG PDF

% CZ anotace + klíčová slova
\begin{abstractCZ}
text českého abstraktu
\end{abstractCZ}

\begin{keywordsCZ}
klíčová slova v češtině
\end{keywordsCZ}

% EN anotace + klíčová slova
\vspace{2cm}

\begin{abstractEN}
text anglického abstraktu
\end{abstractEN}

\begin{keywordsEN}
keywords in english
\end{keywordsEN}

\clearpage

% Poděkování (volitelné)
\begin{acknowledgement}
text poděkování (volitelné)
\end{acknowledgement}

% Obsah (odtud se zobrazují čísla stránek)
\TULthesisTOC{}

% Seznam ilustrací / tabulek (volitelné dle obsahu práce)
\listoffigures%
\vspace{10em}
\listoftables%
\clearpage
\listoflistings%
\clearpage

% Seznam zkratek, značek a symbolů
\begin{abbreviations}
\textbf{TUL} & Technická univerzita v~Liberci \\
\end{abbreviations}

\cleardoublepage%

% ==========================================================================
% VLASTNÍ PRÁCE (kapitoly)
% ==========================================================================

% Úvod (bez čísla v obsahu)
\chapter*{Úvod}
\addcontentsline{toc}{chapter}{Úvod}
text úvodu

\cleardoublepage%

\chapter{Teoretická část}
text teoretické části

\section{Analýza a zhodnocení současného stavu}
text analýzy a zhodnocení současného stavu

\section{Literární rešerše}
text literární rešerše

\cleardoublepage%

\chapter{Praktická část}
text praktické části

\section{Návrh řešení}
text návrhu řešení

\section{Ekonomické zhodnocení}
text ekonomického zhodnocení

\section{Implementace / Realizace}
text implementace / realizace

\cleardoublepage%

% Závěr (bez čísla v obsahu)
\chapter*{Závěr}
\addcontentsline{toc}{chapter}{Závěr}
text závěru

\cleardoublepage%

% ==========================================================================
% BIBLIOGRAFIE
% ==========================================================================

% VARIANTA A: Automaticky pomocí biblatex (doporučeno, skryto dokud nejsou citace)
\printbibliography[heading=bibintoc, title=\TULbibname]

% VARIANTA B: Ruční sázení (pokud nepoužíváte biblatex)
% \chapter*{\TULbibname}
% \addcontentsline{toc}{chapter}{\TULbibname}
% text ručně vloženého seznamu literatury

\cleardoublepage%

% ==========================================================================
% PŘÍLOHY
% ==========================================================================

\begin{TULthesisAppendix}

% LOA = list of appendices (seznam příloh)
\TULthesisLOA%

\chapter{Název první přílohy}
text první přílohy

\section{Sekce v příloze}
detailní text

\cleardoublepage%

\chapter{Název druhé přílohy}
text druhé přílohy

\end{TULthesisAppendix}

\end{document}
