% tulpackage/example/example-tularticle.tex
% Obsahuje: Příklad použití třídy tularticle z balíku tulpackage

\documentclass[a4paper,12pt,fonts,nocapnumbers,listingsleft]{./tulpackage/tularticle}

% Zvetseni radkovani
\linespread{1.3}

% Vlastní seznamy s enumitem
\newlist{itemize*}{itemize}{2}
\setlist[itemize*]{itemsep=3pt, parsep=3pt, label=$\bullet$}
\setlist[itemize*,2]{label={--}}

% Nastaveni sazby a zalamovani textu
\sloppy
\lefthyphenmin=2
\righthyphenmin=3
\hyphenpenalty=5000
\tolerance=1500
\emergencystretch=3.5em
\hbadness=2000
% Penalizace osamocených řádků
\widowpenalty=10000
\clubpenalty=10000
% Definice pomocneho prikazu pro ukazky
\newcommand{\cmddemo}[1]{\bigskip\parbox[c]{3.9cm}{\cmd{#1}}\parbox[c]{10cm}{\csname #1\endcsname}\bigskip}

% Správa bibliografie s biblatex-biber
\usepackage[backend=biber]{biblatex}
% \addbibresource{./bibliography/example-tularticle-references.bib}
\addbibresource{./bibliography/example-tularticle-references.bib}

% Nastavení jazyka dokumentu na cestinu
\setdefaultlanguage{czech}

% Nazev PDF v ramci metadat
\hypersetup{pdftitle={TU v Liberci (Příkladový článek)}}

% Uzivatelske informace
\newcommand{\doctitle}{TU v~Liberci \newline(Příkladový článek)}
\newcommand{\name}{Ondřej Wiener}
\newcommand{\phone}{ Tel: neuvedeno}
\newcommand{\mail}{ondrej.wiener@tul.cz}
\newcommand{\location}{Liberec~2025}

% Predani informaci sablone
\TULname{\name}
\TULphone{\phone}
\TULmail{\mail}

% Prikazy specificke pro dokument
\newcommand{\cmdfont}[1]{\texttt{\color{\tulcolor}#1}}
\newcommand{\cmdnoindex}[1]{\cmdfont{\textbackslash{}#1}\index{#1@\textbackslash #1}}
\newcommand{\cmd}[1]{\cmdnoindex{#1}\index{#1@\textbackslash #1}}
\newcommand{\demobox}{\raisebox{-.20ex}{\rule{1em}{1em}}}

% Upravena bibliografie
\newcommand{\custombibliography}{
    \vspace{2mm}
    \section*{Reference}
    \vspace{4mm}
    \printbibliography[heading=none]
}


\begin{document}

\title{\doctitle}
\author{\name}
\date{\today}

\TULfancytitlepage{\doctitle}
{\name}
{\location}

% Preddefinovany obsah
\TULarticleTOC{}

\textit{Následující text pochází ze~článku na
\href{https://cs.wikipedia.org/wiki/Technick\%C3\%A1_univerzita_v_Liberci}{cs.wikipedia.com}
s~poslední aktualizací v~září~2024 a~slouží jako ilustrace pro~použití balíku TULPACKAGE \textendash{}
modifikované verze balíku TUL pro~\LaTeX. Formátování dokumentu a~úpravy balíku byly provedeny
za~krátký časový úsek v~rámci semestrální práce z~počítačové typografie a~nemusí tak být perfektně dotažené.}

\section{Technická univerzita v~Liberci}

Technická univerzita v~Liberci (TUL) je vysoká škola založená roku~1953 ve~městě Liberci.
Univerzita má sedm fakult a~jeden odborný ústav.
Vzdělává se na~ní kolem 6\thinspace{}tisíc studentů.~\cite{golka} % Doporučena úzká mezera za číslem

\section{Historie}

Škola byla zřízena rozhodnutím vlády ze~dne 27.~února~1953 jako \textit{Vysoká škola~strojní}~(VŠS).
Pro~novou školu byla uvolněna budova tehdejšího gymnázia F.~X.~Šaldy v~Hálkově ulici.
Dne 1.~října roku~1953 nastoupilo do~prvních ročníků čerstvě otevřené vysoké školy 259~studentů.
Škola měla tehdy šest kateder, na~kterých působilo 19~pedagogů. Zaměřovala se na~obory typické
pro~severní Čechy: strojírenský, textilní, oděvní, sklářský a~keramický průmysl. Tehdejší studenti
byli ubytováni na~internátě v~Zeyerově ulici.~\cite{golka}

\vspace{1.2em}
\begin{figure}[H]
    \includegraphics[width=0.5\linewidth]{pictures/Liberec-TUL-Rektorat-1.jpg}
    \caption{Nový rektorát a~informační centrum}\label{fig:tul-rektorat}
\end{figure}

\clearpage

Během následujících let se škola rozrůstala nejen o~nové studenty,
ale také o~nové prostory: byly postaveny nové koleje, získána budova bývalé textilní továrny
v~Doubí a~další budovy v~okolí dnešního Studentského náměstí. Roku~1958 dokončilo školu
prvních 121~absolventů slavnostní promocí v~libereckém divadle. Roku~1960 byla škola rozdělena
na~fakultu strojní a~textilní a~stala~se~z~ní \textit{Vysoká~škola~strojní~a~textilní
v~Liberci} (VŠST). Dále byly získány budovy v~Sokolské~ulici (dnešní budova~S), objekt
po~zrušeném pedagogickém institutu v~Komenského ulici (budova~P). S~nárůstem počtu studentů
samozřejmě přestala stačit ubytovací kapacita tehdejších kolejí. Proto byla roku~1977
v~libereckém Starém Harcově zahájena výstavba komplexu šesti kolejních bloků o~kapacitě 2300~lůžek,
nové menzy a~dalších zařízení. Komplex byl ve~své dnešní podobě dokončen roku~1990.~\cite{niznansky}

O~dva roky později (1992) byla získána budova bývalého Stavoprojektu ve~Voroněžské ulici (budova~H)
včetně dočasného sídla Investiční a~poštovní banky nově přestavěného na~univerzitní knihovnu.
Ve~stejném roce získala škola také komplex ve~Vesci sloužící jako koleje a~laboratoře a~v~roce
1996 někdejší Dům politické výchovy na~třídě 1.~máje (budova~K).~\cite{golka}

V~letech 1990--1994 zřídila škola další čtyři fakulty: pedagogickou v~roce~1990, hospodářskou (1992),
architektury (1994) a~mechatroniky a~mezioborových inženýrských studií (1995). Díky takovému růstu,
vlastní výzkumné činnosti a~zahraničním stykům byl škole zákonem č.~192/1994~Sb.\ z~27.~9.~1994
přiznán od~1.~ledna~1995 název \textit{Technická univerzita v~Liberci}.~\cite{zakon192}

\vspace{1.2em}
\begin{figure}[H]
    \includegraphics[width=0.6\linewidth]{pictures/kampus_tul.jpg}
    \caption{Kampus TUL, pohled z~budovy~G}\label{fig:kampus-tul}
\end{figure}

\clearpage

\subsection{Rektoři Technické univerzity v~Liberci}

\begin{itemize*}
    \setlength{\itemsep}{-4pt}
    \item prof.\thinspace{}Ing.\thinspace{}Dr.\thinspace{}techn. Josef~Kožoušek (1953--1961)
    \item doc.\thinspace{}Ing.\thinspace{}Vojtěch~Dráb, CSc. (1961--1966)
    \item akad. Jovan~Čirlič (1966--1969)
    \item prof.\thinspace{}Ing.\thinspace{}Jiří~Mayer, DrSc. (1969--1973)
    \item akad. Jovan~Čirlič (1973--1985)
    \item prof.\thinspace{}RNDr.\thinspace{}Bohuslav~Stříž, DrSc. (1985--1990)
    \item prof.\thinspace{}Ing.\thinspace{}Zdeněk~Kovář, CSc. (1990--1997)
    \item prof.\thinspace{}RNDr.\thinspace{}David~Lukáš, CSc. (1997--2003)
    \item prof.\thinspace{}Ing.\thinspace{}Vojtěch~Konopa, CSc. (2003--2010)
    \item prof.\thinspace{}Dr.\thinspace{}Ing.\thinspace{}Zdeněk~Kůs (2010--2018)
    \item doc.\thinspace{}RNDr.\thinspace{}Miroslav~Brzezina, CSc., dr.\thinspace{}h.\thinspace{}c.\ (od~roku~2018)~\cite{golka}
\end{itemize*}

\clearpage

\section{Fakulty a~univerzitní ústavy}

Univerzita má v~současnosti sedm fakult~\cite{golka,jandova}:
\begin{itemize*}
    \setlength{\itemsep}{-4pt}
    \item \textbf{Fakulta strojní}, založena~1953
    \item \textbf{Fakulta textilní}, založena~1960
    \item \textbf{Fakulta přírodovědně-humanitní a~pedagogická} (FPHP/FP), založena~1990 jako Fakulta pedagogická (FP),
    přejmenována~2008
    \item \textbf{Ekonomická fakulta} (EF), založena~1992 jako Hospodářská fakulta (HF), přejmenována~2009
    \item \textbf{Fakulta umění a~architektury} (FUA/FA), založena~1994 jako Fakulta architektury (FA), přejmenována~2007
    \item \textbf{Fakulta mechatroniky, informatiky a~mezioborových studií} (FMIMS/FM), založena~1995 jako
    Fakulta mechatroniky a~mezioborových inženýrských studií (FMMIS/FM), přejmenována~2008
    \item \textbf{Fakulta zdravotnických studií} (FZS), založena~2016, vznikla z~Ústavu zdravotnických studií (ÚZS),
    který byl založen v~roce~2004
\end{itemize*}

a~jeden univerzitní ústav:
\begin{itemize*}
    \setlength{\itemsep}{-4pt}
    \item \textbf{Ústav pro nanomateriály, pokročilé technologie a~inovace} (CxI), založen~2009
\end{itemize*}

\vspace{1em}
\begin{figure}[H]
    \includegraphics[width=0.9\linewidth]{pictures/budovy_p_b_t_tul.jpg}
    \caption{Budovy P (FP), B (FT) a~H (EF)}\label{fig:budovy-fp-ft-ef}
\end{figure}

Univerzita dále provozuje \textbf{Centrum dalšího vzdělávání} (CDV). Centrum organizuje řadu kurzů dalšího
a~celoživotního vzdělávání, například kurzy pedagogické přípravy nebo jazykové kurzy. Zároveň CDV zajišťuje
univerzitu třetího věku pro~zájemce od~50\thinspace{}let.~\cite{jandova}

\clearpage

\section{Současnost TUL}

V~areálu Husova je menza, informační centrum, studentský klub, pobočka Univerzitní knihovny a~školka.
Univerzitní školka \uv{ŠkaTULka} je~určena pro~48\thinspace{}dětí ve~věku od~3\thinspace{}let,
kterým nabízí přístup Montessori a~Waldorfské pedagogiky.~\cite{jandova}

O~životě na~Technické univerzitě v~Liberci informuje od~roku~2001 zpravodajský časopis \textit{T-UNI online}.
Na~univerzitě dále působí řada studentských organizací včetně studentské unie.~\cite{jandova}

\subsection{Ubytování}

Studentské koleje jsou umístěny ve~Starém Harcově na~ulici 17.~listopadu 587/8. V~letech 2011, 2013
a~2014 zvítězily ve~studentské anketě \uv{Kolej roku}. Součástí areálu jsou dvě~sportovní haly, tělocvičny,
lezecká stěna, sauna, posilovna, minigolf, lanové centrum, fotbalové hřiště a~hřiště na~beach volejbal.~\cite{menzy}

\subsection{Dětská univerzita}

Od~roku~2008 probíhá na~TUL Dětská univerzita, celoroční volnočasové neformální vzdělávání dětí a~mládeže
ve~věku od~6\thinspace{}do~19\thinspace{}let v~oborech: elektrotechnika, fyzika, přírodní vědy, strojírenství,
textilní obory, robotika, programování a~matematika. Výuka kopíruje vysokoškolské studium včetně promoce,
zápočtů a~zpracování závěrečné práce.
~\cite{jandova} % Doporučeny úzké mezery za čísly

\subsection{Sport}

O~sportoviště se~stará Akademické sportovní centrum, které organizuje také fotbalovou, florbalovou,
basketbalovou a~volejbalovou ligu. Na~TUL existuje Volejbalový klub a~Badmintonový klub,
další z~nich se~soustředí v~Univerzitním sportovním klubu Slavia.
~\cite{golka}

\clearpage

\subsection{Univerzitní sportovní klub Slavia Liberec}

Současný Univerzitní sportovní klub Slavia Liberec (USK Slavia Liberec) byl~založen
v~říjnu~1953 Jaroslavem Tyšlem jako Vysokoškolská tělovýchovná jednota
Vysoké školy strojní (VŠTJ Slavia VŠS), která měla původně pět oddílů: odbíjená, basketbal,
kopaná, lyžování a~lední hokej. Ke~31.~prosinci~2016 měl klub 255\thinspace{}členů
v~osmi sportovních oddílech: basketbal, volejbal, lyžování, tenis (od~roku~1955),
horolezectví (od~roku~1961), karate (od~roku~1977), futsal (od~roku~1994) a~florbal
(od~roku~1995).~\cite{golka} % Doporučena úzká mezera za číslem

\clearpage

\section{Základní informace o~univerzitě}

\vspace{-0.6em}
\begin{table}[H]
    \small
    \begin{tabular}{ll}
        \midrule
        Název                               & Technická univerzita v~Liberci                                      \\
        Zkratka                             & TUL                                                                 \\
        Rok založení                        & 1953                                                                \\
        Typ školy                           & veřejná                                                             \\
        Rektor                              & doc.\thinspace{}RNDr.\thinspace{}Miroslav Brzezina, CSc., dr.\thinspace{}h.\thinspace{}c. \\
        Kvestorka                           & Ing.~Martina Froschová                                                                    \\
        Počet studentů bakalářského studia  & 4230                                                                \\
        Počet studentů magisterského studia & 1419                                                                \\
        Počet studentů doktorského studia   & 299                                                                 \\
        Počet akademických pracovníků       & 576                                                                 \\
        Počet fakult                        & 7                                                                   \\
        Sídlo                               & Liberec                                                             \\
        Web                                 & https://www.tul.cz                                                  \\
        \bottomrule
    \end{tabular}
    \caption{Základní informace o~Technické univerzitě v~Liberci~\cite{vyrocnizprava}}\label{tab:zakladni_informace}
\end{table}
\vspace{0.8em}

\section{Vedení univerzity}

\vspace{-0.6em}
\begin{table}[H]
    \small
    \begin{tabular}{ll}
        \toprule
        \textbf{Funkce}                        & \textbf{Jméno}                                           \\
        \midrule
        Prorektor pro rozvoj                   & PhDr.\thinspace{}Ing.\thinspace{}arch.\thinspace{}Lenka Burgerová, Ph.D. \\
        Prorektorka pro zahraniční vztahy      & doc.\thinspace{}Ing.\thinspace{}Kateřina Maršíková, Ph.D.                 \\
        Prorektor pro vědu a~výzkum            & prof.\thinspace{}Dr.\thinspace{}Ing.\thinspace{}Petr Lenfeld             \\
        Prorektor pro informatiku              & doc.\thinspace{}RNDr.\thinspace{}Pavel Satrapa, Ph.D.                     \\
        Prorektor pro vzdělávání a~vnitřní legislativu & prof.\thinspace{}Ing.\thinspace{}Miroslav Žižka, Ph.D.                    \\
        Předseda                               & doc.\thinspace{}Ing.\thinspace{}Vlastimil Hotař, Ph.D.                    \\
        \bottomrule
    \end{tabular}
    \caption{Vedení Technické univerzity v~Liberci~\cite{vyrocnizprava}}\label{tab:vedeni}
\end{table}

\clearpage

\custombibliography{}

\end{document}
