\documentclass[fonts,numbering,twoside,margins]{./tulpackage/tulthesis}
% Tento dokument používá balíčky specifické pro XeLaTeX a lze jej přeložit
% pouze XeLaTeXem. Pokud nemáte instalována použitá (komerční) písma, změňte
% nebo odstraňte příkazy \set...font na následujících řádcích

\newcommand{\verze}{2.1}

% Nastavení jazyka dokumentu na češtinu
\usepackage{polyglossia}
\setdefaultlanguage{czech}

% Vytvoření indexu dokumentu
\usepackage{makeidx}
\makeindex

% Podpora pro Unicode a XeLaTeX specifické funkce
\usepackage{xunicode}
\usepackage{xltxtra}
% Potřebuji neproporcionální kurzívu, kterou Noto Sans Mono zatím nemá
\setmonofont{Roboto Mono}

% Mezery mezi odstavci místo odsazení prvních řádků
\usepackage{parskip}
% Nastavení sazby a zalamování textu
\sloppy
\lefthyphenmin=2
\righthyphenmin=3
\hyphenpenalty=5000
\tolerance=1500
\emergencystretch=3.5em
\hbadness=2000
% Penalizace osamocených řádků
\widowpenalty=10000
\clubpenalty=10000
% Definice pomocného příkazu pro ukázky
\newcommand{\cmddemo}[1]{\bigskip\parbox[c]{3.9cm}{\cmd{#1}}\parbox[c]{10cm}{\csname #1\endcsname}\bigskip}

% Příkazy specifické pro tento dokument
\newcommand{\argument}[1]{{\ttfamily\color{\tulcolor}#1}}
\newcommand{\argumentprom}[1]{\argument{\{\emph{#1}\}}}
\newcommand{\argumentindex}[1]{\argument{#1}\index{#1}}
\newcommand{\prostredi}[1]{\argumentindex{#1}}
\newcommand{\prikazneindex}[1]{\argument{\textbackslash #1}}
\newcommand{\prikaz}[1]{\prikazneindex{#1}\index{#1@\textbackslash #1}}
\newenvironment{myquote}{\begin{list}{}{\setlength\leftmargin\parindent}\item[]}{\end{list}}
\newenvironment{listing}{\begin{myquote}\color{\tulcolor}}{\end{myquote}}
\newcommand{\polozka}[1]{\item[#1]\mbox{}\\}

% Zvýšení řádkování
\linespread{1.3}

% Vlastní seznamy s enumitem
\usepackage{enumitem}
\newlist{itemize*}{itemize}{2}
\setlist[itemize*]{itemsep=3pt, parsep=3pt, label=$\bullet$}
\setlist[itemize*,2]{label={--}}

% Deklarace pro titulní stránku
\TULtitle{Třída tulthesis, verze~\verze}{tulthesis class, version~\verze}
\TULauthor{doc. RNDr. Pavel Satrapa, Ph.D.}

% Nastavení pro bakalářské, diplomové a disertační práce
\TULprogramme{N2612}{Elektrotechnika a informatika}{Electrical engineering and informatics}
\TULbranch{1802T007}{Informační technologie}{Information technology}
%\TULbranch{1802T008}{Nějaký jiný obor}{Some other branch}
\TULsupervisor{doc. RNDr. Pavel Satrapa, Ph.D.}
%\TULconsultant{doc. RNDr. Pavel Satrapa, Ph.D.}
%\TULconsultant{doc. RNDr. Druhý Konzultant, Ph.D.}
%\TULconsultant{doc. RNDr. Třetí Konzultant, Ph.D.}
\TULyear{2024}

% Nastavení pro habilitační práce (odkomentováno pouze při potřebě)
%\TULbranch{}{Technická kybernetika}{Technical cybernetics}
%\TULyear{2021}

\begin{document}


\ThesisStart{male}
%\ThesisStart{zadani-a-prohlaseni.pdf}

\begin{abstractCZ}
Tato zpráva popisuje třídu \thinspace \texttt{tulthesis} \thinspace pro sazbu absolventských prací Technické univerzity v Liberci pomocí typografického systému \LaTeX.
\end{abstractCZ}

\begin{keywordsCZ}
\LaTeX, balík, třída, TUL
\end{keywordsCZ}

\vspace{2cm}

\begin{abstractEN}
This report describes the \thinspace \texttt{tulthesis} \thinspace package for Technical university of Liberec thesis typesetting using the \LaTeX\ typographic system.
\end{abstractEN}

\begin{keywordsEN}
\LaTeX, package, class, TUL
\end{keywordsEN}

\clearpage

\begin{acknowledgement}
Rád bych poděkoval všem, kteří přispěli ke vzniku tohoto díla.
\end{acknowledgement}

% Předdefinovaný obsah
\TULthesisTOC

\begin{abbrList}
\textbf{TUL} & Technická univerzita v~Liberci \\
\textbf{FM} & Fakulta mechatroniky, informatiky a mezioborových studií
Technické univerzity v~Liberci \\
\end{abbrList}

\chapter{Základy použití}

Návody tohoto typu poměrně často obsahují úvod do \LaTeX u jako takového. Já se
s~dovolením omezím jen na popis samotné třídy \argument{tulthesis} a budu
předpokládat, že jste v~otázkách \LaTeX u dostatečně zorientováni. Pokud
to není pravda, snad vám pomůže můj \emph{\LaTeX\ pro pragmatiky}, který
najdete na adrese:

\begin{myquote}
\href{https://www.nti.tul.cz/~satrapa/docs/latex/}{\emph{https://www.nti.tul.cz/\textasciitilde satrapa/docs/latex/}}
\end{myquote}

Třída \argumentindex{tulthesis} pro typografický systém \LaTeX\ slouží pro sazbu
absolventských prací Technické univerzity v~Liberci. Jedná se o~modifikaci
standardní třídy \argument{report}, která byla rozšířena o~konstrukce pro snadnou
sazbu titulní stránky a dalších obvyklých součástí absolventských prací.

Základní použití je jednoduché~-- v~úvodním příkazu
\prikazneindex{documentclass} uvedete jako jméno třídy hodnotu
\argument{tulthesis}. K~argumentům se ještě dostaneme, tím nejvýznamnějším je
samozřejmě fakulta, na níž práce vzniká. Implicitně se sází titulní strana pro
bakalářskou práci. Jedná-li se o~práci diplomovou, přidejte argument
\argument{DP}, pro disertační práci \argument{Dis} a podobně. Dostupné volby
pro jednotlivé typy prací najdete v kapitole~\ref{volby} na
straně~\pageref{volby}. Bakalářská práce Fakulty mechatroniky tedy bude
začínat:

\begin{listing}
\prikazneindex{documentclass[FM]\{tulpackage/tulthesis\}}
\end{listing}

zatímco diplomová práce Fakulty strojní:

\begin{listing}
\prikazneindex{documentclass[FS,DP]\{tulpackage/tulthesis\}}
\end{listing}

V~preambuli pak musíte v~deklaracích uvést popisné informace: jméno práce, kód
a název studijního programu a případně oboru, své jméno a jméno vedoucího. Všechny
předchozí údaje jsou povinné. Kromě nich můžete uvést i rok vzniku práce. Pokud
jej vynecháte implicitně se doplní rok aktuální v~okamžiku překladu. Dalším
nepovinným údajem je identifikační kód práce, pokud je práci přidělen. Slouží
k~tomu příkazy:

\begin{listing}
\prikaz{TULtitle}\argumentprom{název práce česky}\argumentprom{anglicky}\\
\prikaz{TULprogramme}\argumentprom{kód studijního programu}\argumentprom{název česky}\argumentprom{anglicky}\\
\prikaz{TULbranch}\argumentprom{kód studijního oboru}\argumentprom{název česky}\argumentprom{anglicky}\\
\prikaz{TULauthor}\argumentprom{jméno autora}\\
\prikaz{TULsupervisor}\argumentprom{jméno vedoucího}\\
\prikaz{TULyear}\argumentprom{rok}\\
\end{listing}

U habilitačních prací se deklaruje méně údajů, podrobněji viz
část~\ref{habilitace} na straně~\pageref{habilitace}. Po deklaracích už může
přijít \prikazneindex{begin\{document\}} a bezprostředně za ním příkaz
\prikaz{ThesisStart}\argumentprom{rod}, kde \argument{\emph{rod}} má
hodnotu \argument{female}, pokud je autorkou žena, nebo \argument{male},
jestliže práci píše muž. Na hodnotě tohoto povinného parametru závisí rod
použitý v~prohlášení o~autorství. Příkaz \prikaz{ThesisStart} vysází:

\begin{itemize}
\item titulní stránku v odpovídající jazykové verzi,
\item prázdnou stránku, která se nahradí zadáním,
\item prohlášení o~autorství a autorských právech.
\end{itemize}

Následuje typicky abstrakt a klíčová slova ve dvou jazycích, pro něž jsou
k~dispozici prostředí \prostredi{abstractCZ}, \prostredi{keywordsCZ},
\prostredi{abstractEN} a \prostredi{keywordsEN}, poděkování (prostředí
\prostredi{acknowledgement}), obsah (standardní příkaz
\prikazneindex{tableofcontents}), seznam zkratek (prostředí \prostredi{abbrList}) a
pak již vlastní text práce.

Jelikož vychází z~třídy \argument{report}, je základním příkazem pro jeho
strukturování \prikazneindex{chapter}. Pro kratší dokumenty lze volbou
\argument{article} změnit výchozí třídu na \argument{article} a jako základ pro
členění dokumentu pak používat příkazy \prikazneindex{section}.

Typické zahájení diplomové práce na Fakultě mechatroniky pro sazbu \XeLaTeX em
bude proto vypadat nějak takto:

\begin{listing}
\begin{verbatim}
\documentclass[FM,DP]{tulthesis}
\usepackage{polyglossia}
\setdefaultlanguage{czech}

\TULtitle{Nějaká práce}{Some thesis}
\TULprogramme{N2612}{Elektrotechnika a informatika}%
{Electrotechnology and informatics}
\TULbranch{1802T007}{Informační technologie}%
{Information technology}
\TULauthor{Bc. Jiří Chytrý}
\TULsupervisor{prof. Ing. Svatopluk Moudrý, CSc.}
\TULyear{2018}

\begin{document}

\ThesisStart{male}

\begin{abstractCZ}
Český abstrakt
\end{abstractCZ}

\begin{keywordsCZ}
klíčová slova česky
\end{keywordsCZ}

\vspace{2cm}

\begin{abstractEN}
English abstract
\end{abstractEN}

\begin{keywordsEN}
keywords in English
\end{keywordsEN}

\begin{acknowledgement}
Vám poděkování a lásku vám.
\end{acknowledgement}

\tableofcontents

\clearpage

\begin{abbrList}
\textbf{EU} & Evropská unie \\
\end{abbrList}

\chapter{Úvod}

Ve své diplomové práci...
\end{verbatim}
\end{listing}

Dáváte-li přednost pdf\LaTeX u (což byste rozhodně neměli dělat, používejte raději
moderní implementace \XeLaTeX\ nebo Lua\LaTeX), změní se úvodní skupina příkazů na:

\begin{listing}
\begin{verbatim}
\documentclass[FM,DP]{tulthesis}
\usepackage[czech]{babel}
\usepackage[utf8]{inputenc}
\end{verbatim}
\end{listing}

Kompletní příklad použití třídy najdete ve zdrojovém textu tohoto dokumentu,
který je součástí distribuce vizuálního stylu TUL pro \LaTeX.


\chapter{Závislosti a instalace}

Snažil jsem se, aby závislosti třídy byly co nejmenší. Třída samotná kromě
standardních součástí \LaTeX u (třída \argument{report} a balík
\argument{ifthen}) využívá jen balíky \argument{tul}, \argument{tabularray},
\argument{hyperref} a \argument{pdfpages}. Další závislosti ovšem plynou
z~použití balíku \argument{tul}, podrobnosti najdete v~jeho dokumentaci.

Třída je distribuována v~jednom archivu společně s~balíkem \argument{tul} a
nemá proto samostatnou instalaci. Sama je tvořena jen souborem
\emph{tulthesis.cls}, který stačí umístit kamkoli, kde jej vaše instalace
\LaTeX u najde. 

Oficiální distribuční adresou je:

\begin{listing}
\href{https://www.nti.tul.cz/~satrapa/vyuka/latex-tul/}{\emph{https://www.nti.tul.cz/\textasciitilde satrapa/vyuka/latex-tul/}}
\end{listing}


\chapter{Volby}\label{volby}

Činnost balíku lze ovlivňovat různými volbami. Ty lze rozdělit do několika
skupin:

\begin{description}
\item[Vlastní volby] určují zejména typ práce:

\begin{itemize}
\item \argumentindex{DP}~-- diplomová práce,
\item \argumentindex{BP}~-- bakalářská práce (nedělá nic, bakalářská
práce je implicitní hodnotou, je zařazena jen pro zachování symetrie voleb),
\item \argumentindex{Dis}~-- disertační práce,
\item \argumentindex{Teze}~-- teze disertační práce,
\item \argumentindex{Autoref}~-- autoreferát disertační práce,
\item \argumentindex{Hab}~-- habilitační práce,
\item \argumentindex{SP}~-- semestrální práce,
\item \argumentindex{Proj}~-- projekt.
\end{itemize}

Kromě nastavení typu práce jsou k dispozici ještě následující volby:

\begin{itemize}

\item \argumentindex{article}~-- použít pro dokument základní třídu \LaTeX u
\argument{article}. Je vhodná pro kratší dokumenty (např. autoreferát nebo
teze).

\item \argumentindex{report}~-- použít pro dokument základní třídu \LaTeX u
\argument{report} (nedělá nic, toto je výchozí nastavení).

\end{itemize}

\item[Volby balíku \argument{tul}] jsou předány tomuto balíku. Zde uvedu jen
ty, které mají na vzhled práce přímý dopad:

\begin{itemize}
\item \argumentindex{FS},
\argumentindex{FT},
\argumentindex{FP},
\argumentindex{EF},
\argumentindex{FA},
\argumentindex{FM},
\argumentindex{FZS} a
\argumentindex{CXI}~-- určují součást, na které práce vzniká; mají dopad jednak
na titulní stránku, jednak na volbu základní barvy. Neuvedete-li žádnou volbu
součásti, použije se styl celé univerzity.

\item \argumentindex{EN}~-- práce je v angličtině. Projeví se na jazyku titulní
stránky a prohlášení v \prikaz{ThesisStart}.

\item \argumentindex{bw}~-- sází se černobíle, titulní stránka a nadpisy částí
textu nebudou používat barvy. Lze předpokládat, že tato volba bude
u~absolventských prací dost obvyklá.

\item \argumentindex{fonts}~-- použít písma podle vizuální identity TUL. Bez
této volby ponechá výchozí písma \LaTeX u.

\item \argumentindex{sfheadings}~-- nadpisy nejvyšších úrovní sázet
bezserifovým písmem, nikoli písmem TUL Mono.

\end{itemize}

\item[Volby třídy \argument{report} (resp. \argument{article})] se předají této
třídě a mají standardní dopad. V~základu se třída volá s~volbami
\argument{a4paper} a \argument{12pt}, protože práce se standardně sázejí
dvanáctibodovým písmem na papír formátu A4. Tyto dvě volby jsou pevné,
případnými dalšími můžete ovlivnit chování třídy, například nastavit
oboustranný tisk nebo způsob zarovnání matematických vzorců.

\end{description}


\chapter{Příkazy a prostředí}

V~této kapitole najdete přehled příkazů a prostředí poskytovaných třídou
\argument{tulthesis}. Nejsou řazeny abecedně (na to je jich příliš málo), ale
v~pořadí, které odpovídá pořadí jejich použití v~dokumentu.


\section{Příkazy pro preambuli}\label{deklarace}

Cílem této skupiny příkazů je definovat důležité popisné informace o~práci.

\begin{description}

\polozka{\prikaz{TULtitle}\argumentprom{název práce česky}\argumentprom{anglicky}}
definuje název práce. První argument obsahuje jeho českou verzi, druhý verzi
anglickou. Povinný příkaz.

\polozka{\prikaz{TULprogramme}\argumentprom{kód programu}\argumentprom{název
programu česky}\argumentprom{anglicky}} definuje kód a název studijního
programu, v~jehož rámci je práce vytvořena. Kód je pro obě jazykové verze
shodný, proto se zadává samostatným parametrem. Povinný příkaz kromě
habilitačních prací.

\polozka{\prikaz{TULbranch}\argumentprom{kód oboru}\argumentprom{název oboru
česky}\argumentprom{anglicky}} definuje kód a název studijního oboru nebo
specializace, v~jehož rámci je práce vytvořena. Povinný příkaz pro habilitační
práce, kde obsahuje obor habilitačního řízení. Pro ostatní typy prací je
nepovinný, ale může se vyskytnout i vícekrát. Studuje-li student kombinaci
několika oborů (typické pro FP), uvede po jednom příkazu \prikaz{TULbranch} pro
každý obor.

\polozka{\prikaz{TULauthor}\argumentprom{jméno autora}} definuje titul a jméno autora
práce. Povinný příkaz.

\polozka{\prikaz{TULsupervisor}\argumentprom{jméno vedoucího}} definuje titul a jméno
vedoucího práce. Povinný příkaz kromě habilitačních prací.

\polozka{\prikaz{TULconsultant}\argumentprom{jméno konzultanta}} definuje
titul a jméno konzultanta práce, pokud byl určen. Nepovinný příkaz. Lze jej
použít vícekrát, jestliže má práce více konzultantů.

\polozka{\prikaz{TULyear}\argumentprom{rok vytvoření práce}} definuje rok, ve kterém
práce vznikla. Příkaz je nepovinný, pokud jej vynecháte, doplní se při překladu
automaticky aktuální rok. Přesto doporučuji jej použít, jinak by při případném
pozdějším překladu mohla vzniknout iluze, že práce vznikla v~pozdějším roce,
než ve skutečnosti.

\polozka{\prikaz{TULthesisType}\argumentprom{typ práce česky}\argumentprom{anglicky}}
umožňuje definovat jiný typ práce, než na které pamatují volby balíku
\argument{BP}, \argument{DP} a další. V~prvním argumentu uveďte český
název typu práce, ve druhém jeho anglický ekvivalent. Například pro výzkumnou
zprávu by preambule obsahovala:

\begin{listing}
\begin{verbatim}
\TULthesisType{Výzkumná zpráva}{Research report}
\end{verbatim}
\end{listing}

\end{description}

\clearpage

\section{Příkazy a prostředí pro tělo práce}

Tyto příkazy a prostředí generují viditelný výstup a sází jednotlivé
součásti, především pro úvodní partie práce.

\begin{description}

\polozka{\prikaz{ThesisStart}\argumentprom{rod}} v závislosti na parametru
buď vygeneruje, nebo vloží standardní zahájení práce~-- titulní list, zadání a
prohlášení o~autorství a autorských právech. Parametr \argument{\emph{rod}} má
v základní podobě hodnotu \argument{female} nebo \argument{male} a určuje, zda
bude prohlášení uvedeno v~rodě ženském nebo mužském. Interně se uvnitř volá
níže uvedená trojice příkazů.

Pro bakalářské a diplomové práce jsou v současnosti úvodní stránky generovány
IS/STAG. Chcete-li je použít, předejte příkazu jako parametr název PDF souboru.
\prikaz{ThesisStart} pak jednoduše vloží jeho obsah. Podrobnosti najdete v
části~\ref{stag} na straně~\pageref{stag}.

\polozka{\prikaz{ThesisTitle}\argumentprom{jazyk}} vygeneruje titulní list práce.
Informace čerpá z~výše uvedených deklaračních příkazů \prikaz{TULtitle} a spol.
Argument určuje, jaká jazyková verze se má vysázet. K dispozici je česká
(hodnota \argument{CZ}) a anglická (\argument{EN}) verze titulního listu. Pokud
použijete \prikaz{ThesisStart}, nemusí vás zajímat~-- je určen pro situace, kdy
vám standardní zahájení práce nevyhovuje a chcete je nějakým způsobem upravit.

\polozka{\prikaz{Assignment}} vygeneruje prázdný list, který se v~práci nahradí
originálem jejího zadání. Má dva účely: upomíná, abyste nezapomněli vložit
zadání, a zajišťuje správné číslování stránek (list se zadáním se počítá mezi
stránky práce, přestože na něm číslo není uvedeno). Pokud použijete
\prikaz{ThesisStart}, nemusí vás zajímat~-- je určen pro situace, kdy vám
standardní zahájení práce nevyhovuje a chcete je nějakým způsobem upravit.

\polozka{\prikaz{Declaration}\argumentprom{rod}} vygeneruje prohlášení o~autorství a
autorských právech. Parametr \argument{\emph{rod}} má hodnotu
\argument{female} nebo \argument{male} a určuje, zda bude prohlášení uvedeno
v~rodě ženském nebo mužském. Pokud použijete \prikaz{ThesisStart}, nemusí vás
zajímat~-- je určen pro situace, kdy vám standardní zahájení práce nevyhovuje a
chcete je nějakým způsobem upravit.

\polozka{\prostredi{abstractCZ}} je prostředí pro sazbu českého abstraktu práce.
Nedělá mnoho~-- vysází český název práce, nadpis „Abstrakt“ a zúží řádek,
protože abstrakty obvykle bývají krátké a na plnou šířku řádku nevypadají
dobře. Je-li váš abstrakt dlouhý, použijte nepovinný argument \argument{wide},
který vynechá zúžení a sází abstrakt na celou šířku řádku:

\begin{listing}
\begin{verbatim}
\begin{abstractCZ}[wide]
Široký text abstraktu.
\end{abstractCZ}
\end{verbatim}
\end{listing}

Volba \argument{wide} je k dispozici pro všechna čtyři prostředí zajišťující
sazbu abstraktu a klíčových slov. Používejte ji konzistentně~-- u všech nebo u
žádného.

\polozka{\prostredi{abstractEN}} je prostředí pro sazbu anglického abstraktu práce.
Od \argument{abstractCZ} se liší jen anglickým názvem práce a nadpisem.

\polozka{\prostredi{keywordsCZ}} je prostředí pro sazbu českých klíčových slov.
Podobně jako abstrakt vysází nadpis „Klíčová slova“ a zúží řádek. Opět je k
dispozici volba \argument{wide}, která řádky nezužuje. Předpokládá se, že
klíčová slova následují vždy za abstraktem.

\polozka{\prostredi{keywordsEN}} je prostředí pro sazbu anglických klíčových slov.
Od \argument{keywordsCZ} se liší jen anglickým nadpisem.

\polozka{\prostredi{acknowledgement}} je prostředí pro poděkování. Opět se chová
stejně~-- vysází nadpis „Poděkování“ a sází svůj obsah do užších řádků, pokud
nepoužijete nepovinný argument \argument{wide}.

\polozka{\prostredi{abbrList}} je prostředí pro sazbu seznamu zkratek. Obsah
seznamu zkratek je koncipován jako tabulka se dvěma sloupci. Zkratku od popisu
proto oddělte znakem \argument{\&} a popis zakončete
\argument{\textbackslash\textbackslash}. Použití vypadá nějak takto:

{\color{\tulcolor}
\begin{verbatim}
\begin{abbrList}
CSS  & Cascading Style Sheets \\
HTML & HyperText Markup Language \\
PHP  & PHP: Hypertext Preprocessor \\
W3C  & World Wide Web Consortium \\
\end{abbrList}
\end{verbatim}}

Komentář k seznamu zkratek najdete v~části~\ref{zkratky}.

\end{description}


\chapter{Závěrečné poznámky}

Několik poznámek a povzdechů závěrem.


\section{Úvodní stránky generované IS/STAG}\label{stag}

V současnosti jsou úvodní stránky bakalářských a diplomových prací (titulní
strana, zadání a autorské prohlášení) generovány IS/STAG. Na začátek práce se
jednoduše vloží. V tom případě se nemusíte zatěžovat deklaracemi popsanými v
kapitole~\ref{deklarace} na straně~\pageref{deklarace}. Z nich stačí použít jen
\prikaz{TULtitle}, protože název práce se používá v abstraktu.

Pro vložení PDF souboru vygenerovaného z IS/STAG jednoduše předejte jeho jméno
jako parametr příkazu \prikaz{ThesisStart}. Příkaz pak sám nedělá nic, jen
místo sebe vloží obsah zadaného souboru. Postupujte následovně:

\begin{enumerate}
\item Vygenerujte z IS/STAG úvodní stránky práce a uložte je do PDF souboru,
například \emph{zahajeni.pdf}.

\item Na začátek práce vložte příkaz \prikaz{ThesisStart} a jako jeho argument
uveďte název příslušného souboru, tedy:

\prikaz{ThesisStart}\argument{\{zahajeni.pdf\}}

\item Práci znovu přeložte. V místě příkazu \prikaz{ThesisStart} by měl být
vložen obsah externího PDF souboru. Pokud chybí, pravděpodobně jste udělali
chybu v názvu souboru.

\end{enumerate}

\clearpage

\section{Habilitační práce}\label{habilitace}

Třída \argument{tulthesis} původně vznikla pro bakalářské a diplomové práce a
postupně se rozšiřoval sortiment podporovaných typů prací. Jsou vesměs dost
konzistentní, zadávají se pro ně obdobné údaje a generují podobné stránky.
Výrazně odlišné jsou práce habilitační~-- nemají vedoucího ani zadání, liší se
autorské prohlášení a podobně.

Pokud použijete volbu \argumentindex{Hab}, velmi se zjednoduší úvodní
deklarace. Stačí uvést název, autora, obor a rok vzniku práce. Obor
habilitačního řízení nemívá přiřazen kód a může chybět i jeho název v
angličtině. Příslušné parametry příkazu \prikaz{TULbranch} ale musí být
uvedeny, i když jsou prázdné.

Zahájení habilitační práce může vypadat například nějak takto:

\begin{listing}
\begin{verbatim}
\documentclass[FM,Hab]{packege_tul/tulthesis}
\TULtitle{Stochastické řízení}{Stochastic control}
\TULbranch{}{Technická kybernetika}{Technical cybernetics}
\TULauthor{Ing. Jan Řidič, Ph.D.}
\TULyear{2021}
\end{verbatim}
\end{listing}


\section{Kombinace s~dalšími balíky}

Snažil jsem se spíše o~minimalistický přístup, aby třída nedělala víc než musí
a pokud možno se nepletla do cesty různým balíkům. Její činnost, která by
mohla mít dopady na kompatibilitu s~dalšími balíky, se omezuje na:

\begin{itemize}
\item Nastavení velikosti papíru a okrajů. K~tomu se používají standardní
prostředky \LaTeX u~-- volba \argument{a4paper} a rozměry
\prikazneindex{oddsidemargin} a spol.

\item Nastavení zápatí stránky prostřednictvím balíku \argument{fancyhdr}.
Chcete-li zasahovat do záhlaví nebo zápatí stránek, dělejte to pomocí příkazů
tohoto balíku.

\item Změna formátu nadpisu kapitol pomocí balíku \argument{titlesec}. Opět,
pokud vám nevyhovuje, využijte ke změně nástroje poskytované tímto balíkem,
konkrétně příkaz \prikazneindex{titleformat}.

\end{itemize}

Nasazení dalších balíků a nástrojů pro změnu vzhledu dokumentu či rozšíření
možností sazby by nemělo stát nic v~cestě. Nechávám to na vás.

Například jsem rezignoval na nastavení konkrétního písma pro sazbu práce,
protože písmo a způsob jeho volby závisí jak na písmech dostupných ve vašem
systému, tak na konkrétní variantě \LaTeX u, kterou používáte. Pokud používáte
\XeLaTeX, máte k~dispozici balík \argument{fontspec}. Sázíte-li pdf\-\LaTeX\-em
(což byste už dělat neměli, připravujete se o vymoženosti moderních
implementací, jako jsou \XeLaTeX nebo Lua\LaTeX), můžete sáhnout po některém ze
standardních balíků typu \argument{palatino} či \argument{newcent}.

\clearpage

\section{Seznam zkratek}\label{zkratky}

Ve verzi~2.1 bylo prostředí \argument{abbrList} pro seznam zkratek přepracováno
s využitím balíku \argument{tabularray}. Nyní je implementováno jako dlouhá~--
tedy potenciálně vícestránková~-- tabulka (prostředí \argument{tblr}) se dvěma
sloupci, kde v prvním jsou zkratky a ve druhém jejich významy. Mělo by
vyhovovat pro valnou většinu případů.

Pokud se délky jednotlivých zkratek výrazně liší, tabulková varianta
pravděpodobně nebude vypadat dobře~-- levý sloupec bude široký a za krátkými
zkratkami budou velké mezery. V tom případě raději nevyužijte připravené
prostředí \argument{abbrList} a implementujte si seznam zkratek vhodnějším
způsobem, například prostřednictvím standardního prostředí
\argument{description}. Tedy něco jako:

\begin{listing}
\begin{verbatim}
\section*{Seznam zkratek}\phantomsection
\addcontentsline{toc}{chapter}{Seznam zkratek}

\begin{description}
\item[HTML] HyperText Markup Language
\item[CSS] Cascading Style Sheets
\end{description}
\end{verbatim}
\end{listing}

\clearpage

\renewcommand{\indexname}{Přehled příkazů, prostředí a voleb}
\phantomsection\addcontentsline{toc}{chapter}{\indexname}

\begingroup
\titlespacing{\chapter}{0pt}{10pt}{*4} % Adjust only for the index
\renewcommand{\indexname}{Přehled příkazů, prostředí a voleb}
\phantomsection\addcontentsline{toc}{chapter}{\indexname}
\printindex
\endgroup % Restore chapter spacing to what it was before


\end{document}
