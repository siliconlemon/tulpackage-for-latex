% tulpackage/modules/formatting.tex
% Description: Formatting for Titles, TOC, Title Page, and Listings environments.

% Oddělovač v textu (svislá čára ve fakultní barvě)
\newcommand{\oddel}{{\color{\tulcolor}\rule{1pt}{1.7ex}}}
% Styl a barva nadpisů
\RequirePackage[sf, bf]{titlesec}
% Font nadpisů (Uses \TULHeadFont defined in fonts.tex)
\newcommand{\TUL@headfnt}{\TULHeadFont}

% Definice centrální barvy pro obsah a seznam obrázků/tabulek
\newcommand{\TUL@TOCtextcolor}{black}

% Balík pro tečkování kapitol, centrální barvu textu v obsahu
\RequirePackage{tocloft}
% Nastavení fontu pro nadpis obsahu
\renewcommand{\cfttoctitlefont}{%
    \color{\titlecolor}\TUL@headfnt\LARGE\raggedright%
}

% Deklarace mezer mezi položkami obsahu
\newlength{\TUL@TOC@skip}

\@ifundefined{chapter}{%
    % Nastavení mezer v obsahu pro dokumenty bez kapitol (např. article)
    \setlength{\cftbeforetoctitleskip}{2mm}%
    \setlength{\cftaftertoctitleskip}{10mm}%
    \renewcommand{\cftsecleader}{\cftdotfill{\cftdotsep}}%
    \setlength{\TUL@TOC@skip}{8pt}%
}{%
    % Nastavení mezer v obsahu pro dokumenty s kapitolami (např. book, report)
    \setlength{\cftbeforetoctitleskip}{0mm}%
    \setlength{\cftaftertoctitleskip}{6mm}%
    \renewcommand{\cftchapleader}{\cftdotfill{\cftdotsep}}%
    \setlength{\TUL@TOC@skip}{4pt}%
    \renewcommand{\cftbeforechapskip}{\TUL@TOC@skip}%
}

% Použití společné mezery pro všechny položky na úrovni sekcí
\renewcommand{\cftbeforesecskip}{\TUL@TOC@skip}%
\renewcommand{\cftbeforesubsecskip}{\TUL@TOC@skip}%
\renewcommand{\cftbeforesubsubsecskip}{\TUL@TOC@skip}%

% Fonty v obsahu
\renewcommand{\cftpartfont}{\color{\TUL@TOCtextcolor}}
\renewcommand{\cftloftitlefont}{\color{\TUL@TOCtextcolor}\LARGE\bfseries}
\renewcommand{\cftfigfont}{\color{\TUL@TOCtextcolor}}
\renewcommand{\cfttabfont}{\color{\TUL@TOCtextcolor}}

% Balík pro vynucenou centrální barvu hypertextu v obsahu
\RequirePackage{etoolbox}
\pretocmd{\l@chapter}{\hypersetup{linkcolor=\TUL@TOCtextcolor}}{}{}
\pretocmd{\l@section}{\hypersetup{linkcolor=\TUL@TOCtextcolor}}{}{}
\pretocmd{\l@subsection}{\hypersetup{linkcolor=\TUL@TOCtextcolor}}{}{}

% Balík pro vynucenou centrální barvu textu a hypertextu v seznamu obrázků/tabulek
\pretocmd{\l@figure}{\hypersetup{linkcolor=\TUL@TOCtextcolor}}{}{}
\pretocmd{\l@table}{\hypersetup{linkcolor=\TUL@TOCtextcolor}}{}{}

% Seznam zkratek
\newenvironment{abbrList}{
    \phantomsection%
    \ifthenelse{\equal{\TUL@nguage}{EN}}
    {\section*{\large List of abbreviations}\vspace{-1em}}
    {\section*{\large Seznam zkratek}\vspace{-1em}}
    \thispagestyle{plain}
    % Reset šablon pro tabularray
    \DefTblrTemplate{contfoot-text}{default}{}
    \DefTblrTemplate{conthead-text}{default}{}
    \DefTblrTemplate{contfoot}{default}{}
    \DefTblrTemplate{conthead}{default}{}
    \DefTblrTemplate{caption}{default}{}
    \DefTblrTemplate{capcont}{default}{}
    \noindent\tblr[long]{column{2}={co=1}}
}{
    \endtblr%
    \clearpage
}

% Titulní stránka
\newcommand{\create@uthb@x}[2]{%
    \vspace{20mm}
    {\Large #1}\\
    \vspace{5mm}
    {\large #2}\par
}

% Výška textu s příponou "pt" pro násobení v řádkování
\newlength{\f@ntsize}
\setlength{\f@ntsize}{\f@size pt}

% Úvodní strana TUL
\newcommand{\TULtitlepage}[3]{{%
            \thispagestyle{TULheaderonly}
            \mbox{}
            \vfill
            \vfill
            \vfill
            \bgroup%
            \raggedright%
            \color{\tulcolor}\huge\bfseries\TULmono\setlength{\baselineskip}{2.5\f@ntsize} #1\par
            \create@uthb@x{#2}{#3}
            \egroup%
            \vfill%
            \newpage%
        }}

% Fancy úvodní strana TUL
\newcommand{\TULfancytitlepage}[3]{{%
            \thispagestyle{TULheaderonly}
            \pagecolor{\tulcolor}\darkTULbg%
            \mbox{}
            \vfill
            \vfill
            \vfill
            \bgroup%
            \raggedright%
            \color{white}\huge\TULFancyFont\setlength{\baselineskip}{2.5\f@ntsize} #1\par
            \create@uthb@x{#2}{#3}
            \egroup%
            \vfill
            \newpage
            \pagecolor{white}
        }}

% Definice formátování nadpisů (makro, bude voláno v layout.tex)
\newcommand{\setup@TUL@titleformats}{%
    \def\setupTitleFormat##1##2##3##4{%
        \ifthenelse{\boolean{TUL@numbering}}{%
            \titleformat{##1}{\color{\titlecolor}\TUL@headfnt##2\raggedright}{##3}{1em}{}%
        }{%
            \titleformat{##1}{\color{\titlecolor}\TUL@headfnt##2\raggedright}{}{0pt}{}%
        }%
    }
    % CHAPTER
    \ifcsname chapter\endcsname % chktex 1
        \setupTitleFormat{\chapter}{\LARGE}{\thechapter}{}%
    \fi
    % SECTION
    \ifcsname section\endcsname % chktex 1
        \setupTitleFormat{\section}{\Large}{\thesection}{}%
    \fi
    % SUBSECTION
    \ifcsname subsection\endcsname % chktex 1
        \setupTitleFormat{\subsection}{\large}{\thesubsection}{}%
    \fi
    % SUBSUBSECTION
    \ifcsname subsubsection\endcsname % chktex 1
        \setupTitleFormat{\subsubsection}{\normalsize}{\thesubsubsection}{}%
    \fi
    % PARAGRAPH - no numbering
    \ifcsname paragraph\endcsname % chktex 1
        \titleformat{\paragraph}{\color{\titlecolor}\TUL@headfnt\normalsize\raggedright}{}{0pt}{}%
    \fi
}

% === LISTINGS / BLOKY KÓDU ===
\usepackage{framed}
% listoflistings - seznam bloků kódu
\newcommand{\listoflistings}{%
    \clearpage
    \phantomsection%
    \ifthenelse{\equal{\TUL@nguage}{EN}}
    {\section*{\large List of Listings}\@mkboth{List of Listings}{List of Listings}}
    {\section*{\large Seznam listingů}\@mkboth{Seznam listingů}{Seznam listingů}}
    \thispagestyle{plain}
    {%
        \hypersetup{linkcolor=black}%
        \@starttoc{lol}%
    }%
}
% Formatting for the list lines:
\newcommand{\l@tullisting}{\@dottedtocline{0}{0pt}{2.3em}}

% listingcaption - titulek pro bloky kódu
\newcounter{tullisting}
\newcommand{\listingcaption}[1]{%
    \par\vspace{0pt}
    \refstepcounter{tullisting}
    \addcontentsline{lol}{tullisting}{\protect\numberline{\thetullisting}#1}%
    \noindent{Listing \thetullisting:} #1
    \par\vspace{8pt}
}
% myquote - základní blok citátu, použit pro bloky kódu
\newenvironment{myquote}{\begin{list}{}{\setlength\leftmargin\parindent}\item[]}{\end{list}}
% listing - standardní blok kódu, velký text
\newenvironment{listing}{\begin{myquote}\color{\tulcolor}}{\end{myquote}}
% slisting - zmenšený blok kódu, menší text
\newenvironment{slisting}{%
    \vspace{16pt}
    \begin{myquote}
        \footnotesize
        \color{\tulcolor}
        \def\FrameCommand{\hspace{16pt}\hspace{0.6pt}\hspace{10pt}}
        \MakeFramed{}{\advance\hsize-\width{}\FrameRestore}
        }{%
        \endMakeFramed{}
    \end{myquote}
    \vspace{0pt}
}
% linelisting - blok kódu s čárou, menší text
\newenvironment{linelisting}{%
    \vspace{16pt}
    \begin{myquote}
        \footnotesize
        \color{\tulcolor}
        \def\FrameCommand{\hspace{28pt}{\color{\tulcolor}\vrule width 0.6pt} \hspace{10pt}}
        \MakeFramed{}{\advance\hsize-\width{}\FrameRestore}
        }{%
        \endMakeFramed{}
    \end{myquote}
    \vspace{0pt}
}
